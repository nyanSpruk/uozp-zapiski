\chapter{Projekcije in zmanjšanje dimenzionalnosti podatkov}

Modeli, ki jih gradimo v strojnem učenju, povzemajo podatke tako, da v nekem formalnem zapisu predstavijo glavne vzorce, ki so te podatke oblikovali. V primeru večdimenzionalnih podatkov nas tako na primer zanima, ali bi podatke lahko popisali z manjšim številom spremenljivk. Z izborom značilk smo se ukvarjali že v prejšnjem poglavju, a nas bo tu zanimalo, če lahko podatke predstavimo z novimi značilkami, ki bi podatke lahko predstavili bolj kompaktno, pri tem odkrili kakšno zanimivo preslikavo starih v nove značilke, ter morda celo dosegli, da je novih značilk izjemno malo. Recimo, dve, tako da lahko vse primere izrišemo v razsevnem diagramu in tam s pomočjo vizualizacije razmišljamo o njihovih podobnosti in strukturi prostora primerov. Ljudje smo vizualna bitja in nam izris kart primerov lahko intuitivno pove več kot pa recimo matematične enačbe, ki morda te povezujejo.

Začnimo s primerom. Na sliki~\ref{f:torbice} so ženske torbice različnih proizvajalcev. Z uporabo že zgrajenih globokih mrež za razvrščanje slik lahko slike torbic pretvorimo v vektorje. Mreža Inception v3 nam na svojem predzadnjem nivoju na primer za vsako sliko vrne vektor dolžine 2048, oziroma sliko popiše z 2048 atributi. S postopki, ki jih bomo opisali v tem poglavju, lahko take opise zmanjšamo in predstavimo torbice v nizko dimenzionalnem prostoru. Slika~\ref{f:torbice-mds} tako prikazuje predstavitev primerov v dvodimenzionalnem prostoru, kjer je razvidno, da ena Hermesova torbica bolj podobna torbicam ostalih proizvajalcev, in da se je Dior pri eni od torbic zgledoval po torbicah Chanela. Morda. Vsekakor bi ta opažanja morali preveriti, a je zanimivo, kako hitro nas vizualizacije navdahnejo z idejami. Prav zato so vizualizacije v odkrivanju znanj iz podatkov tako pomembne in zato so pomembne tudi tehnike zmanjšanja dimenzionalnosti in odkrivanja nizko dimenzionalnih projekcij podatkov.

\begin{figure}[htbp]
\centering{
\includegraphics[width=12cm]{slike/torbice.png}
\caption{Ženske torbice različnih proizvajalcev.}
\label{f:torbice}}
\end{figure}

\begin{figure}[htbp]
\centering{
\includegraphics[width=8cm]{slike/torbice-mds.png}
\caption{Predstavitev ženskih torbic s slike~\ref{f:torbice} v dvodimenzionalnem prostoru.}
\label{f:torbice-mds}}
\end{figure}

\section{Metoda glavnih komponent}

Prva, morda tudi najbolj znana metoda za zmanjšanje dimenzij in odkrivanje zanimivih projekcij podatkov je metoda glavnih komponent. Predpostavimo, da imamo dano matriko učnih primerov $X\in\mathbb{R}^{m\times n}$, kjer je $m$ število primerov in $n$ število atributov. Primer $x^{(i)}\in X$, torej $i$-ti primer v učni množici primerov, bo tako opisan z $n$ atributi $x\in\mathbb{R}^{n}$, za katere bomo tu privzeli, da so njihove vrednosti realna števila. V splošnem iščemo projekcije podatkov tako, da atributni zapis primerov z $n$ atributi nadomestimo z atributnim zapisom z $k$ novimi atributi tako, da je $k\ll n$ in da nam novi atributi povedo čim več o primerih iz učne množice.

Pričnimo s $k=1$, torej s projekcijo v eno samo dimenzijo. Dogovorimo se, da bo naša projekcija linearna in da bomo vrednost novega atributa tvorili kot linearno kombinacijo originalnih atributov. Premico, na katero bomo projicirali podatke, označimo z enotskim vektorjem $u_1$, za katerega torej velja $u_1^\tr u_1=1$.

Naj bo $x^{(i)}$ primer iz učne množice. Njegova pravokotna projekcija na premico je skalar, $u_1^\tr x^{(i)}\in \mathbb{R}$. Sama vrednost tega skalarja nam ne pove prav dosti; potrebovali bi neko referenčno točko, od katere bi merili razdalje naših projekcij. To referenčno točko lahko pridobimo kot projekcijo središča podatkov. Naj bo $\bar{x}$ središčna točka podatkov:
%
$$ \mean{x}=\frac{1}{m}\sum_i x^{(i)}$$
%
Projekcija središčne točke na projekcijsko ravnino je $u_1^\tr \mean{x}$. Primeri v učni množici odstopajo od središčne točke, eni bolj, drugi manj. Projekcijsko premico bi želeli zato poiskati tako, da so ta odstopanja čim bolj zajeta, torej da so projekcije primerov na premico, ki jo določa $u_1$, čim bolj razpršene. Razpršenost merimo s povprečnim kvadratnim odstopanjem, torej povprečnim kvadratom razdalje projekcije in projekcije središčne točke podatkov. Tako določeni razpršenosti pravimo tudi varianca točk v projekciji:
%
$$ \Var(u_1^\tr X^\tr)=\frac{1}{m}\sum_{i=1}^m \big(u_1^\tr x^{(i)}-u_1^\tr\mean{x}\big)^2 $$

Naš cilj je torej poiskati tak projekcijski vektor $u_1$, ki maksimizira varianco projekcije $\Var(u_1^\tr X^\tr)$. Poigrajmo se malce z enačbo za varianco projekcije:
%
\begin{equation}
  \begin{split}
    \Var(u_1^\tr X^\tr)
    & = \frac{1}{m}\sum_{i=1}^m \big(u_1^\tr x^{(i)}-u_1^\tr\mean{x}\big)^2 \\
    & = \frac{1}{m}\sum_{i=1}^m \big(u_1^\tr x^{(i)} u_1^\tr x^{(i)} - 2 u_1^\tr x^{(i)} u_1^\tr \mean{x} + u_1^\tr \mean{x} u_1^\tr \mean{x} \big)
  \end{split}
\end{equation}
%
Upoštevajmo, da je $u_1^\tr x$ skalarni produkt dveh vektorjev in da je transponirana vrednost skalarja enaka temu skalarju. Torej je na primer $u_1^\tr x=(u_1^\tr x)^\tr=x^\tr u_1$. Z upoštevanjem tega se nam zgornji izraz primerno poenostavi:
%
\begin{equation}
  \begin{split}
    \Var(u_1^\intercal X^\tr)
    & = \frac{1}{m}\sum_{i=1}^m u_1^\tr \big(x^{(i)} {x^{(i)}}^\tr - 2 x^{(i)} \mean{x}^\tr + \mean{x}~\mean{x}^\tr \big) u_1 \\
    & = u_1^\tr \big( \frac{1}{m}\sum_{i=1}^m (x^{(i)}-\mean{x}) (x^{(i)}-\mean{x})^\tr\big) u_1
  \end{split}
\end{equation}

Produkt $(x^{(i)}-\mean{x}) (x^{(i)}-\mean{x})^\tr$ je matrika velikosti $n\times n$. Matriko sestavljajo produkti vseh možnih parov elementov vektorja $(x^{(i)}-\mean{x})$, torej, v prvi vrstici produkt med prvim elementom in njim samim, pa med prvim in drugim elementom, pa prvim in tretjim in tako naprej. Povprečna vrednost takih matrik, ki jih dobimo s seštevanjem matrik za vsakega od primerov, je kovariančna matrika! Prejšnji stavek smo končali s klicajem zato, ker je kovarianca znan in mnogokrat uporabljan koncept v statistiki in nam pove, kako sta povezani dve naključni spremenljivki, oziroma v našem primeru dva atributa. Označimo kovariančno matriko s ${\rm S}$:
\begin{equation}
  {\rm S} = \frac{1}{m}\sum_{i=1}^m (x^{(i)}-\mean{x}) (x^{(i)}-\mean{x})^\tr
\end{equation}
V našem primeru imamo $n$ atributov, kar pomeni, da je kovariančna matrika dimenzije ${\rm S}=\mathbb{R}^{n\times n}$.

Zapišimo še enkrat dobljeno enačbo za varianco:
\begin{equation}
  \Var(u_1^\intercal X^\tr) = u_1^\tr {\rm S} u_1
\end{equation}
%
Spomnimo, da je naš namen poiskati projekcijski vektor $u_1$ pri katerem je zgornja varianca maksimalna. Za vektor $u_1$ zahtevamo, da je to enotni vektor (sicer maksimizacija zgornje enačbe ne bi imela smisla, saj bi to dosegli z neskončno dolgim vektorjem $u_1$). Maksimizacijo variance, kjer iščemo ustrezen vektor $u_1$ z omejitvijo $u_1^\tr u_1=1$ rešimo z uporabo Lagrangeovih multiplikatorjev. Maksimiziramo torej izraz:
%
\begin{equation}
  f(u_1) = u_1^\tr {\rm S} u_1 + \lambda_1(1-u_1^\tr u_1)
\end{equation}
%
V maksimumu bo odvod zgornje enačbe enak nič:
\begin{equation}
  \frac{\partial f(u_1)}{\partial u_1}={\rm S} u_1 - \lambda_1 u_1=0
\end{equation}
%
kar pomeni,
\begin{equation}
  {\rm S} u_1=\lambda_1 u_1
\end{equation}
%
Tu je ${\rm S}$ matrika katere vrednosti poznamo, $u_1$ je vektor, ki ga iščemo, $\lambda_1$ pa skalar. Poznano? Seveda! Zgornjemu pogoju ustreza le tak $u_1$, ki je lastni vektor matrike ${\rm S}$. Ampak korelacijska matrika ima več lastnih vektorjev. Kateri med njimi je pravi? Množimo zgornjo enačbo z $u_1^\tr$:
%
\begin{equation}
  u_1^\tr {\rm S} u_1 = u_1^\tr \lambda_1 u_1=\lambda_1 u_1^\tr u_1=\lambda_1
\end{equation}
%
Na levi je izraz za našo varianco $\Var(u_1^\intercal X^\tr)$, ki jo skušamo maksimizirati. Vemo, da mora biti $u_1$ lastni vektor kovariančne matrike. Skladno z zgornjo enačbo pa sedaj tudi vemo, da je to lastni vektor z najvišjo pripadajočo lastno vrednostjo.

V smeri vektorja $u_1$ se lega naših primerov $X$ v $n$-dimenzionalnem prostoru torej najbolj spreminja, njihova varianca je v tej smeri največja. Z $u_1^\tr X^\tr$ dobimo pravokotno projekcijo vsakega od primerov na to os. S to projekcijo, oziroma legi v smeri $u_1$, smo pojasnili največjo varianco primerov. Kaj še ostane? Vse, kar je pravokotno na smer $u_1$. Največjo varianco na pravokotni hiperravnini pa je ravno v smeri drugega lastnega vektorja, saj je ta pravokoten na $u_1$, to je vektorja, ki ima drugo največjo lastno vrednost. Pojasnjen delež te variance je $\lambda_2$, skupaj pa s pravokotno projekcijo na prvi in drugi lastni vektor pojasnimo $\lambda_1+\lambda_2$ variance množice primerov.

Lastni vektorji kovariančne matrike nam določajo nov koordinatni
sistem, kjer so koordinate, po vrsti, tiste, po katerih se lege točk v
večdimenzionalnem prostoru najbolj spreminjajo, oziroma je njihova
varianca največja. Nove koordinate, urejene skladno z lastnimi vrednostmi vektorjev, imenujemo komponente primerov v novem, transformiranem sistemu. Ker nas bodo zanimale samo te z visokimi deleži razložene variance jih bomo imenovali {\em glavne komponente}.

\begin{figure}[htbp]
\centering{
\includegraphics[width=8cm]{slike/scree-diagram.pdf}
\caption{Graf odvisnosti razložene variance od števila glavnih komponent (zgornja črta) za podatke o torbicah iz slike~\ref{f:torbice}.}
\label{f:scree}}
\end{figure}

V splošnem nas zanima, koliko najvišje rangiranih dimenzij v novem
koordinatnem sistemu zares potrebujemo za opis podatkov. Pri tem moramo
seveda določiti, kakšen je naš ciljni delež pojasnjene variance. Tipično
smo zadovoljni, če izbrano število komponent pojasni vsaj $80\%$ skupne variance. Ker skupno varianco poznamo (enaka je vsoti kvadratov razdalj do
centra), je potrebno torej le določiti, koliko začetnih urejenih lastnih
vrednosti moramo sešteti, da njihova vsota predstavlja želeni delež razložene variance. Primer grafa, ki kaže odvisnost deleža razložene variance glede na število glavnih komponent kaže slika~\ref{f:scree}.

Pristop glavnih komponent se mnogokrat uporablja tudi samo za namene vizualizacije, kjer primere, opisane z mnogimi atributi, želimo predstaviti v dvodimenzionalni ravnini. Pri tem se moramo seveda zavedati, da prvi dve komponenti razložita morda le majhen del celotne variance.

Za predstavitev podatkov z glavnimi komponentami je potrebno podatke najprej osrediščiti in nato poiskati kovariančno matriko. Lastni vektorji kovariančne matrike so bazni vektorji novega, transformiranega sistema, njihove lastne vrednosti pa povedo, kakšen delež variance nam razloži posamezna komponenta.

Še praktičen razmislek: v primerih velikega števila atributov bo računanje vseh lastnih vektorjev in vrednosti zamudno. Še posebej, če nas zanima samo projekcija podatkov v dvodimenzionalni prostor. Prvi lastni vektor kovariančne matrike $M$ lahko rajši (hitreje) določimo numerično, s potenčno metodo. Vzamemo nek naključni vektor (npr. $x$), ga transformiramo z matriko $M$ (torej, izračunamo $x\leftarrow M x$), in to ponavljamo do konvergence. Ob vsakem koraku tako dobljeni $x$ normaliziramo, torej, $x\leftarrow \frac{M x}{\norm{M x}}$. Na ta način dobimo lastni vektor. Kako izračunamo pripadajočo lastno vrednost?
\begin{equation}
  M u = \lambda u
\end{equation}

\begin{equation}
  u^\tr M u = u^\tr \lambda u = \lambda u^\tr u = \lambda
\end{equation}

Kako določimo naslednjo komponento? Ena od možnosti je, da uporabimo ravnokar izračunani lastni vektor, nanj projiciramo podatke, te vrednosti odštejemo od podatkov (torej dobimo projekcijo na lastnemu vektorju pravokotno hiperravnino) in za te podatke ponovno izračunamo kovariančno matriko ter s potenčno metodo njej lastni vektor z največjo lastno vrednostjo. V primeru, da potrebujemo več komponent, postopek ustrezno ponovimo.

Za primer, ko nas zanima samo projekcija podatkov v ravnino lahko potenčno metodo namesto na vektorju izvedemo na sistemu dveh vektorjev $U$. Vektorja matrike $U$ morata biti pravokotna. To dosežemo tako, da po vsakem koraku potenčne metode uporabimo Gram-Schmidtovo ortogonalizacijo.

\section{Večrazredno lestvičenje}

Pri zmanjšanju dimenzij nam je lahko cilj, da ohranimo razdalje med posameznimi primeri. Tu predpostavimo, da za vsak par primerov znamo med njimi oceniti razdaljo. Se pravi, če imamo primere predstavljene v atributnem zapisu, potem uporabimo recimo Evklidsko razdaljo ali pa kosinusno podobnost. Lepota tehnike, ki jo bomo predstavili v tem razdelku je, da nas ne zanima, kako je bila podobnost oziroma razdalja med primeri ocenjena. Lahko smo na vhodu na primer imeli tekstovne dokumente, za katere razdaljo recimo izračunamo s kakšno kompresijsko tehniko. Ali pa filme, za katere razdaljo izračunamo z ozirom na njihovo gledanost v spletni sposojevalnici. Ali pa čivke, ki jih primerjamo med sabo glede na to, kdo jih je všečkal.

Naj bo razdalja med primeroma enaka $\delta_{ij}$. Sedaj te primere vložimo, recimo, v dvodimenzionalno projekcijo tako da vsakemu primeru $i$ pripišemo koordinati $\vect{\theta}^{(i)}=(\theta_1^{(i)},\theta_2^{(i)})$. Želeli bi, da se v njej razdalje ohranjajo, to je, da je projekcijska razdalja, ki naj bo $d_{ij}$, čimbolj podobna originalni razdalji med primeroma. Razdalja primerov v projekcijskem prostoru je:
$$\delta_{ij}=\norm{{\vect{\theta}}^{(i)}-\vect{\theta}^{(j)}}$$

Računalniško izvedbo algoritma, ki poišče tako vložitev, imenujemo večrazsežnostno lestvičenje ({\em multidimensional scaling}, MDS). Formalno ciljno funkcijo problema zastavimo tako, da določimo energijo sistema ali napetost ({\em stress}) kot, recimo,
$$J(\mathbf{X}) = \sum_{i\ne j} (d_{ij} - \delta_{ij})^2,$$
kjer je $\mathbf{X}$ trenutni razpored točk, $d_{ij}$ in $\delta_{ij}$ pa trenutna projekcijska in želena razdalja med primeroma $i$ in $j$. Večrazsežnostno lestvičenje poišče razpored $\mathbf{X}$ z najmanjšo napetostjo $J(\mathbf{X})$ oziroma najmanjšo vrednostjo kriterijske funkcije.

Kako minimizirati takšno funkcijo? Analitično ne bo šlo. Za to bi bilo potrebno odvesti $\sigma(\mathbf{X})$ po vseh koordinatah $\theta_1^{(i)}$ in $\theta_2^{(i)}$ in poiskati ničle:
\begin{eqnarray}
  \frac{\partial\sigma({\mathbf{X}})}{\partial \vect{\theta_k}} & = &
  \sum_{j\ne i} \frac{\partial\sigma({\mathbf{X}})}{\partial d_{ij}}\frac{d_{ij}}{\partial \vect{\theta_k}} \nonumber \\
  & = & - \sum_{j\ne i} 2\left(\sqrt{(\theta_1^{(i)}-\theta_1^{(j)})^2+(\theta_2^{(i)}-\theta_2^{(j)})^2}-\delta_{ij}\right) \frac{(\theta_k^{(i)}-\theta_k^{(j)})}{\sqrt{(\theta_1^{(i)}-\theta_1^{(j)})^2+(\theta2^{(i)}-\theta_2^{(j)})^2}} \nonumber \\
& = & - \sum_{j\ne i} 2\left(d_{ij}-\delta_{ij}\right) \frac{(\theta_k^{(i)}-\theta_k^{(j)})}{d_{ij}} \label{eq:mds-grad}
\end{eqnarray}
Pri stotih točkah bi dobili sistem z dvesto takšnimi enačbami; ideji o direktnem napadu se torej raje odpovemo.

Drug, izvedljiv postopek je gradientna metoda: kot pri dosedanjih pristopih z gradientnim spustom (recimo, linearna regresija) izračunamo gradient funkcije kriterijske funkcije $J(\mathbf{X})$. To smo pravzaprav storili že zgoraj. Izberemo si poljuben začetni razpored točk, vstavimo koordinate v odvode, dobimo gradient (vektor odvodov) in se pomaknemo v smeri nasprotni gradientu, se pravi, vse točke pomaknemo v smer, kamor jih potiskajo odvodi. Gradientna metoda bi delovala, vendar ne prav dobro. Bila bi počasna in velikokrat bi nas (lahko) pripeljala le v lokalni minimum.

V svetu optimizacijskih problemov je gradientna metoda le metoda grobe sile, ki jo bomo uporabili, če se ne moremo spomniti ničesar boljšega. Eden od boljših postopkov je ``majorizacija''. Označimo funkcijo, katere minimum iščemo, z $f(x)$. Recimo, da si znamo izmisliti neko drugo funkcijo $g(x, y)$, ki je stalno večja ali enaka $f(x)$ (torej, za vsak $y$ velja $g(x, y)\ge f(x)$ pri vsakem $x$), v točki $g(x, x)$ pa velja $g(x, x)=f(x)$. Funkcija $g(x, y)$ naj bo takšna, da znamo dobro poiskati vrednost $x$, pri kateri doseže minimum. Zdaj lahko počnemo tole: izmislimo si začetni $x_0$. Pri njem, vemo, velja $g(x_0, x_0)=f(x_0)$. Minimiziramo funkcijo $g$ in dobimo nov, boljši $x$ - označimo ga z $x_1$. Vrednost $g(x_1, x_0)$ je manjša ali enaka vrednosti $g(x_0, x_0)$, saj smo minimizirali $g$ po prvem argumentu. Po drugi strani, spet po definiciji funkcije $g$, velja $f(x_1) \le g(x_1, x_0)$. Imamo torej
$$f(x_1) \le g(x_1, x_0) \le g(x_0, x_0) = f(x_0)$$
Postopek ponovimo z $x_1$, da pridelamo naslednji, še boljši $x_2$.

Takšna optimizacija je, tako kot gradientna metoda še vedno iterativna, vendar hitrejša. Koliko hitrejša, je odvisno od tega, kako dobro $g(x, y)$ smo poiskali; $g(x, y)$ se mora čim tesneje prilegati $f(x)$, poleg tega pa jo moramo biti zmožni čim boljše optimizirati.

Algoritem večdimenzionalnega lestvičenja, ki deluje na ta način, se imenuje SMACOF \angl{scaling by majorizing a complicated function}. SMACOF je hitrejši od gradientne metode, dela večje korake in se manjkrat zatakne v lokalnih ekstremih, zato je uporabnejši za (realistične) primere, v katerih je točk veliko.

Za izračun napetosti lahko namesto gornje, naivne formule uporabljamo tudi takšno, ki uteži prispevke parov (predpišemo lahko, za katere pare si še posebej želimo, da bi razdalja ustrezala pravi), poleg tega pa lahko napetost posameznega para normiramo tako, da jo delimo z želeno ali s trenutno razdaljo.

Slika~\ref{f-zoo-mds} kaže zemljevid živali. Razdalje med pari živali so izračunane kot evklidske razdalje med atributi, ki jih opisujejo. Točke, ki predstavljajo živali, smo obarvali glede na vrsto -- vijolični so sesalci, rdeče ptice, oranžno žuželke. Povezani so pari živali, ki so si med seboj najbolj podobne.

\begin{figure}[tbp]
\begin{center}
\includegraphics[width=10cm]{slike/zoo-mds.png}
\caption{Zemljevid živali (podatkovna zbirka Zoo). Vrsta živali je označena z barvo. Nekatere oznake so zaradi prekrivanj neberljive in bi razvojnike programa potrebno opozoriti, da v vizualizacijo vključijo optimizacijo postavitve oznak.}
\label{f-zoo-mds}
\end{center}
\end{figure}

Čeprav MDS ni uporabljal podatkov o vrstah živali, so različne vrste na zemljevidu dobro ločene. Vidimo tudi nekaj podobnosti med živalmi: morske sesalce je MDS postavil v bližino rib; dvoživke so blizu rib in nevretenčarjev, predvsem morskih; žuželke so pomešane s ptiči.

Vidimo tudi težavo: žuželke so razdeljene v dve skupini. MDS se je očitno znašel v lokalnem ekstremu, saj bi morale ose in čebele k ostalim žuželkam, vendar bi jih moral optimizacijski postopek peljati okrog ptičev. To se zgodi kar pogosto in pomagamo si tako, da podatke nekoliko potresemo, naključno premaknemo točke, ali pa začnemo celotno optimizacijo znova.

Večrazsežnostno lestvičenje je torej postopek, ki vizualizira podatke v obliki zemljevida. Podatki so opisani s seznamom primerov in razdalj med njimi; lastnosti primerov niso potrebne in jih ne znamo upoštevati (razen, če iz le-teh računamo razdalje, kot smo storili pri živalih). Zemljevid je nezanesljiv v tem smislu, da bodo različne naključne začetne postavitve vodile v različne končne slike, a vsaka od njih nam lahko nudi drugačen pogled na podatke. Prav tako je nezanesljiv položaj posameznih točk; dogaja se, da je točka obtičala v lokalnem minimumu, čeprav v resnici ne sodi na to mesto. Takšne točke lahko prepoznamo, če zemljevidu dodamo povezave med najbolj podobnimi pari. Obenem nam takšne povezave pomagajo brati sliko. MDS ne sestavlja gruč, pač pa lahko gruče razberemo sami.

\section{Stohastična vložitev sosedov}

Je lahko še kaj boljšega za vložitev primerov v dvorazsežni prostor kot večrazredno lestvičenje? Oziroma, kaj bi bilo lahko sploh narobe s tehniko MDS? Problem MDS-a je, da se ta osredotoča tako na primere, ki so si v originalnem prostoru blizu, kot tudi na primere, ki so si daleč. MDS skuša ohraniti razdaljo med primeri ne glede na to, kako velika je. Velikokrat pa nas v vizualizacijah zanima samo to, da so primeri, ki so si med seboj podobni, skupaj tudi v projekcijskem prostoru. Torej, ohranjanje razdalj med primeri, ki so si med seboj drugačni, nas ne zanima, ali pa nas, recimo, zanima veliko manj kot ohranjanje bližine med seboj podobnih primerov.

Metoda SNE \angl{stochastic neighbor embedding}, ki smo jo tu poslovenili v stohastično vložitev sosedov, temelji na verjetnostni oceni podobnosti dveh primerov. Za primera $i$ in $j$ oziroma njuni vektorski predstavitvi $x^{(i)}$ in $x^{(j)}$ določimo verjetnost $p_{j|i}$, da bi za primer $i$ izbrali soseda $j$, če bi sosede izbirali skladno z Gaussovo verjetnostno porazdelitvijo, ki je centrirana v $x^{(i)}$:
\begin{equation}
  p_{j|i}=\frac{\exp(-\norm{x^{(i)}-x^{(j)}}^2/2\sigma_i^2)}
  {\sum_{k\neq i}\exp(-\norm{x^{(i)}-x^{(k)}}^2/2\sigma_i^2)}
\end{equation}
kjer je $\sigma_i$ varianca Gaussove porazdelitve za točko $i$. Ker nas zanimajo samo modeliranje podobnosti med različnimi točkami, je $p_{i|i}$ enaka nič. Zanima nas vložitev teh primerov v projekcijski prostor, kjer bo lega primerov kot pri MDS določena z vektorji $\vect{\theta}^{(i)}$ in bomo podobno kot zgoraj, le tokrat za projekcijski prostor, sosednost predstavili z verjetnostjo $q_{j_i}$.

Označimo verjetnostne porazdelitve preko vseh sosedov z za primer $i$ in za originalni prostor z $P_i$ in za projekcijski prostor z $Q_i$. Naša želja je, da bi bili ti porazdelitvi med sabo čim bolj podobni. Ker ti dve verjetnostni porazdelitve favorizirajo bližnje primere oziroma so za te verjetnosti visoke, podobnost porazdelitev $P_i$ in $Q_i$ pomeni, da je projekcija ohranila bližnje primere za primer $i$. Mera, ki pove, kako zvesto $q_{j|i}$ sledi $p_{j|i}$ je Kullback-Leibler-jeva divergenca, katere vsoto po primerih želimo minimizirati. Kriterijska funkcija je zato:
\begin{equation}
  J(\vect{\Theta}) = \sum_i KL(P_i\ ||\ Q_i)=\sum_i\sum_j p_{j|i}\log\frac{p_{j|i}}{q_{j|i}}
\end{equation}
Zaradi nesimetričnosti Kullback-Leibler-jeve divergence bodo napake obravnavane nesimetrično: cena za v projekciji oddaljene točke, ki bi sicer morale biti blizu (majhen $q$ in velik $p$) bo večja kot za točke, ki so bližje v projekciji, a bi sicer morale biti daleč (velik $q$ in majhen $p$). Metoda SNE primarno ohranja sosednost.

Med parametri kriterijske funkcije nam je ostal nedoločen še $\sigma_i$, parameter verjetnostne porazdelitve za primer $i$. Metoda SNE ga določi skladno z lokalno gostoto primerov tako, da je za gostejše okolice $\sigma_i$ primerno manjši.

Da bi določili vložitev moramo poiskati lege primerov $\vect{\theta}^{(i)}$ v projekcijskem prostoru. To lahko vnovič storimo z gradientnim sestopom:
%
\begin{equation}
\nonumber
\pd{\vect{\Theta}}{\vect{\theta}^{(i)}}=
  2\sum_j(p_{j|i}-q_{j|i}+p_{i|j}-q_{i|j})(\vect{\theta}^{(i)}-\vect{\theta}^{(j)})
\end{equation}
%
Gradient (popravek) bo večji pri večji razliki med $p$ in $q$ (prvi člen v produktu) med primeroma in bo potekal v smeri vektorja med primeri (drugi člen produkta).

Danes je predvsem znana ``popravljena'' verzija metode SNE, ki jo imenujemo $t$-SNE \angl{t-distributed stochastic neighbor embedding}. V njej Gaussovo porazdelitev za projekcijski prostor zamenjajo s Studentovo $t$-porazdelitvijo, ki ima daljši rep. S to zamenjavo $t$-SNE dovoli, da so primeri, ki so v srednji okolici v originalnem prostoru lahko daleč stran v projekcijski ravnini. Tehnika namesto pogojnih verjetnosti uporablja združene verjetnosti, in simetrično oceno sosednosti za primere v originalnem prostoru,
%
\begin{equation}
  \nonumber
  p_{ij}=\frac{\exp(-\norm{x^{(i)}-x^{(j)}}^2/2\sigma^2)}
  {\sum_{k\neq l}\exp(-\norm{x^{(k)}-x^{(l)}}^2/2\sigma^2)}
\end{equation}
%
ter za projekcijski prostor
\begin{equation}
  \nonumber
  q_{ij}=\frac{(1+\norm{x^{(i)}-x^{(j)}}^2)^{-1}}
  {\sum_{k\neq l}(1+\norm{x^{(k)}-x^{(l)}}^2)^{-1}}
\end{equation}
%
Tudi tu porazdelitvi primerjamo s Kullback-Leiblerjevo divergenco,
\begin{equation}
  \nonumber
  J(\vect{\Theta}) = \sum KL(P\ ||\ Q)=\sum_i\sum_j p_{ij}\log\frac{p_{ij}}{q_{ji}}
\end{equation}
za katero poiščemo gradient:
%
\begin{equation}
  \nonumber
  \pd{\vect{\Theta}}{\vect{\theta^{(i)}}} = 4\sum_j(p_{ij}-q_{ij})
\frac{(\vect{\theta^{(i)}}-\vect{\theta^{(j)}})}{(1+\norm{\vect{\theta^{(i)}}-\vect{\theta^{(j)}}}^2)}.
\end{equation}
Zgornja enačba je po strukturi presenetljivo podobna enačbi~\ref{eq:mds-grad}  za gradient pri večrazrednem lestvičenju, kar je glede na drugačen način izpeljave nekoliko presenetljivo, glede na intuitivno razumevanje cilja obeh projekcij pa morda sploh ne. Glavna razlika obeh metod je seveda ocenjevanje razdalj oziroma pojmovanje, oziroma bolje uteževanje sosednosti. Če pri večrazrednem lestvičenju razdalja ni utežena, je pri metodi SNE in izvedenkah stopnja sosednosti eksponencialno padajoča funkcija.

Metoda $t$-SNE je v zadnjih letih postala izjemno popularna. Njen problematičen del je predvsem v parametrizaciji (iskanje prave vrednosti za parametre Gaussove distribucije) in v relativni počasnosti metode oziroma pripadajočega gradientnega pristopa.
